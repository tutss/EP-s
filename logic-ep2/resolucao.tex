%% Article type
\documentclass[a4paper]{article}
\usepackage[a4paper,top=3cm,bottom=2cm,left=3cm,right=3cm,marginparwidth=1.75cm]{geometry}

%% Useful packages
\usepackage{amsmath, amsfonts, amssymb}
\usepackage{enumerate}
\usepackage[utf8]{inputenc}

%% Title
 \title{\textbf{EP2 - MAC236}}
 \author{Artur M. R. dos Santos\\
IME-USP, São Paulo\\
N\textsuperscript{o} USP: $10297734$}
 \date{}

%% Start
\begin{document}
\maketitle

\section{Introdução}
O EP2 foi documentado por meio de {\LaTeX}, e cada seção seguinte irá representar a
resolução de um exercício. Utilizei o TexMaker para edição.

\section{Exercícios}

\subsection{Resposta:}
Iremos modelar o problema com $\mathcal{M}$:{\
\begin{itemize}
\item $\mathcal{A} =$ é o domínio infinito.
\item $\mathcal{C}$ = \{x, y\} , onde $\mathcal{C}$ representa o conjunto de constantes 
do domínio $\mathcal{A}$.
\item $\mathcal{F} = \{s\}$ , onde $\mathcal{F}$ representa o conjunto das funções
do domínio, e s é um símbolo funcional binário. 
\\ A função s\textsuperscript{{$\mathcal{M}$}} representa o "sucessor de".  
\item $\mathcal{P} = \{ s(x, y) \in \mathcal{A} \: \vert$ y é sucessor de x \} , onde $\mathcal{P}$
representa o conjunto dos predicados de $\mathcal{A}$.
\end{itemize}
A fórmula será da forma: $\:$ \emph{$\forall x \exists y (\: s(x, y) \:)$} , para todo x, existe um y
que é seu sucessor. Fica evidente que esta fórmula funciona apenas para um conjunto infinito, pois nele,
sempre haverá um elemento sucessor, diferentemente de um conjunto finito, em que não é possível garantir
que sempre haja um próximo elemento. 

\subsection{Resposta:}
Irei utilizar a assinatura genérica $\Sigma$ = $(\mathcal{P}, \mathcal{C}, \mathcal{F})$, com $\mathcal{P} = \{ =_{2}$\}, o predicado de igualdade entre dois elementos, representando os predicados, $\mathcal{C}$ representando as constantes e $\mathcal{F}$ representando as funções.
\begin{enumerate}[a)]
\item A fórmula será : $\:$ $\exists x_{1}, x_{2} \Big[\forall y (\: x_{1} = y \lor x_{2} = y \:)$ $\wedge$ $(\: x_{1} \neq x_{2} \:) \Big]$
\item A fórmula será : $\:$ $\exists  x_{1}, x_{2}, x_{3}, x_{4} \Big[ \forall  y (\: x_{1} = y \lor x_{2} = y \lor x_{3} = y \lor x_{4} = y \:)$ $\wedge$ $(\: x_{1} \neq x_{2} \:)$ $\wedge$ $(\: x_{1} \neq x_{3} \:)$ $\wedge$ $(\: x_{1} \neq x_{4} \:)$ 
$\wedge$ $ (\: x_{2} \neq x_{3} \:)$ $\wedge$ $ (\: x_{2} \neq x_{4} \:)$ $\wedge$ $ (\: x_{3} \neq x_{4} \:)\Big] $
\end{enumerate}

\subsection{Resposta:}

Irei utilizar a assinatura $\Sigma$ = $(\mathcal{P}, \mathcal{C}, \mathcal{F})$, com $\mathcal{P} = \{ =_{2}$\}, o predicado de igualdade entre dois elementos, representando os predicados, $\mathcal{C}$ representando as constantes e $\mathcal{F}$ representando as funções. Nosso modelo $\mathcal{M}$ possui 2n elementos, e teremos $x_1, x_2, \ldots, x_{2n} \in \mathcal{C}:$ \\\\
A fórmula será: $\exists x_1, x_2, \ldots , x_{2n} \Bigg[ \forall y \Big( \: \bigvee_{i=1}^{2n} x_i = y \Big) \land ( \: \bigwedge_{i=1}^{2n} \bigwedge_{j=i+1}^{2n-1} \; x_i \neq x_j ) \Bigg]$


\end{document}